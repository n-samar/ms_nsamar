\section{Introduction}

\subsection{Fully Homomorphic Encryption}

A significant portion of computing workloads are now executed in the cloud.
However, executing a program in the cloud involves users to trust their chosen Cloud Service Provider (CSP)
with access to the data their computations require. Even if a user's trusted CSP is not maliciously inspecting their data,
the CSP is suceptible to being hacked by other actors who can now access user data.
For sensitive information such as health records or financial information, this risk might
be unacceptable.

Fully Homomorphic Encryption (FHE) provides a promising solution to this problem.
FHE systems provide a cryptographically secure method, though computationally expensive, to perform computations on
encrypted data \cite{gentry:stoc09:fhe, brakerski:itcs12:bgv, cheon:asiacrypt2017:ckks}.
Specifically, FHE allows for a ``server" to evaluate an arbitrary arithmetic
circuit of additions and mulitplications on encrypted inputs.
This means that the ``user", who provides the encrypted inputs and a set of public keys, 
does not need to trust the server with access to their data.

Writing an FHE program is quite different from writing a program in a standard
programming language like C++ or Python. 
FHE programs must be arithmetic \emph{circuits}, meaning that 
data dependent branches, and therefore loops, are impossible to express.
The intuitition behind this is that a program cannot branch on data it cannot
even decrypt. This means that not all standard ``plaintext" (unencrypted) programs
can be efficiently translated into FHE programs. Importantly, many numeric
programs including linear algebra, logistic regression, and neural networks
\emph{can} be written as arithmetic circuits 
\cite{halevi:acc14:algo-helib, han:iaai19:he-logreg, dowlin:icml16:cryptonets}.
These algorithms are also particularly appealing candidates for homomorphic
encryption because they are often run on sensitive data.

Unfortunately, all FHE programs have significant computational overheads 
over standard plaintext programs.
Several software libraries \cite{palisade, helib, seal}
implement state of the art FHE cryptosystems, but they all
suffer from several order-of-magnitude slowdowns over plaintext computations.
These massive slowdowns make running homomorphically encrypted programs
infeasible in practice.
Hardware acceleration seems like a promising approach to closing the
performance gap between FHE and plaintext computations.

\subsection{Prior Work}

To this end, several FPGA accelerators have been proposed, each showing
substantial speedups over CPU-based software implementations
\cite{riazi:asplos20:heax, roy:hpca19:he-fpga}.
However, these FPGA designs present several drawbacks. 
FPGAs cannot emulate all of the logic required to execute a full FHE program
simultaneously, and are therefore limited to accelerating only limited chunks
of the overall computation at once.
Additionally, an FPGA design must be tailored to a specfic FHE cryptosystem and
arithmetic circuit, leaving limited flexibility at runtime. 
Finally, FPGAs are very power hungry and have very limited clock speeds.

Furthermore, all existing accelerators fall short of performing
\emph{Fully} Homomorphic Encryption, and instead only implement
\emph{Somewhat} Homomorphic Encryption (SHE). The key difference is that
SHE circuits can only be of limited depth; a specific value cannot be the
result of more than a fixed number of operations, otherwise it will become
too noisy and therefore undecryptable. 
To turn a SHE scheme into a FHE scheme,
we need a \emph{bootstrapping} operation which ``resets" an encrypted value's
noise. Bootstrapping operations have been implemented in FHE software
libraries, but to our knowledge, no one has attempted to accelerate
bootstrapping in hardware. Supporting bootstrapping would allow an accelerator
to support more complex linear algebra computations, deeper neural networks,
and more training iterations for logistic regression.
