\chapter{Introduction}\label{sec:intro}

Despite massive efforts to improve the security of computer systems, security
breaches are only becoming more frequent and damaging, as more sensitive data
is processed in the
cloud~\cite{malekos-smith:csis20:hidden-costs-cybercrime,ibm20:breach-cost-report}.
Current encryption technology is of limited help, because servers must decrypt
data before processing it. Once data is decrypted, it is vulnerable to
breaches.

Fully Homomorphic Encryption (FHE) is a class of encryption schemes that
address this problem by enabling \emph{generic computation on encrypted data}.
\autoref{fig:overview} shows how FHE enables secure offloading of computation.
The client wants to compute an expensive function $f$ (e.g., a deep learning
inference) on some private data $x$. To do this, the client encrypts $x$ and
sends it to an untrusted server, which computes $f$ on this encrypted data
\emph{directly} using FHE, and returns the encrypted result to the client. FHE
provides ideal security properties: even if the server is compromised,
attackers cannot learn anything about the data, as it remains encrypted
throughout.

\figOverview

FHE is a young but quickly developing technology. First realized in
2009~\cite{gentry09}, early FHE schemes were about 10$^9$ times slower than
performing computations on unencrypted data. Since then, improved FHE schemes
have greatly reduced these overheads and broadened its
applicability~\cite{albrecht:hesg18:standard,peikert2016decade}. FHE has
inherent limitations---for example, data-dependent branching is impossible,
since data is encrypted---so it won't subsume all computations. Nonetheless,
important classes of computations, like deep learning
inference~\cite{cheon:ictaci17:homomorphic,dathathri:pldi19:chet,dathathri:pldi20:eva},
linear algebra, and other inference and learning
tasks~\cite{han:aaai19:logistic} are a good fit for FHE. This has sparked
significant industry and government investments~\cite{ibm,intel,dprive} to
widely deploy FHE.

Unfortunately, FHE still carries substantial performance overheads: despite
recent advances~\cite{dathathri:pldi19:chet, dathathri:pldi20:eva,
roy:hpca19:fpga-he, brutzkus:icml19:low, polyakov:17:palisade}, FHE is still
10,000$\times$ to 100,000$\times$ slower than unencrypted computation when
executed in carefully optimized software. Though this slowdown is large, it can
be addressed with hardware acceleration: \emph{if a specialized FHE accelerator
provides large speedups over software execution, it can bridge most of this
performance gap and enable new use cases.}

For an FHE accelerator to be broadly useful, it should be programmable, i.e.,
capable of executing arbitrary FHE computations. While prior work has proposed
several FHE accelerators, they do not meet this goal. Prior FHE
accelerators~\cite{cousins:hpec14:fpga-he,cousins:tetc17:fpga-he,doroz:tc15:accelerating-fhe,roy:hpca19:fpga-he,riazi:asplos20:heax,turan:tc20:heaws}
target individual FHE operations, and miss important ones that they leave to
software. These designs are FPGA-based, so they are small and miss the data
movement issues facing an FHE ASIC accelerator. These designs also
overspecialize their functional units to specific parameters, and cannot
efficiently handle the range of parameters needed within a program or across
programs.

\paragraph{Harnessing opportunities and challenges in FHE:}
We identify three key characteristics of FHE that drive our architectures:

\noindent \textbf{\emph{(1) Complex operations on long vectors:}}
FHE encodes information using very large vectors, several thousand elements
long, and processes them using modular arithmetic. Therefore, an FHE accelerator should employs \emph{vector
processing} with \emph{wide functional units} tailored to FHE operations to
achieve large speedups. The challenge is that two key operations on these
vectors, the Number-Theoretic Transform (NTT) and automorphisms, are not
element-wise and require complex dataflows that are hard to implement as vector
operations. To tackle these challenges, we specialize the NTT units and present
the first vector implementation of an automorphism functional unit.

\noindent \textbf{\emph{(2) Regular computation:}}
FHE programs are dataflow graphs of arithmetic operations on vectors. All
operations and their dependences are known ahead of time (since data is
encrypted, branches or dependences determined by runtime values are
impossible). An accelerator can exploit this by adopting \emph{static
scheduling}: in the style of Very Long Instruction Word (VLIW) processors, all
components have fixed latencies and the compiler is in charge of scheduling
operations and data movement across components, with no hardware mechanisms to
handle hazards (i.e., no stall logic). With this approach, an accelerator can
issue many operations per cycle with minimal control overheads; combined with
vector processing, an effective FHE accelerator can issue tens of thousands of
scalar operations per cycle.

\noindent \textbf{\emph{(3) Challenging data movement:}}
In FHE, encrypting data increases its size (typically by at least 50$\times$);
data is grouped in long vectors; and some operations require large amounts
(tens of MBs) of auxiliary data. Thus, we find that data movement is \emph{the
key challenge} for FHE acceleration: despite requiring complex functional
units, in current technology, limited on-chip storage and memory bandwidth are
the bottleneck for most FHE programs. Therefore, the primary goal of an FHE
accelerator should be to
minimize data movement. To address these issues, we:
\begin{compactenum}
\item Use a large and explicitly managed on-chip memory hierarchy.
\item Decouple data movement. This allows us to hide access latencies by
    loading data far ahead of its use.
\item Have the scheduler focus on maximizing operand reuse, which reduces
    memory bandwidth pressure.
\item Build relatively \emph{few functional units with extremely high
    throughput}, rather than lower-throughput functional units as in prior
    work. This \emph{reduces the amount of data that must reside on-chip
    simultaneously}, allowing higher reuse.
\end{compactenum}

F1 leverages all of these insights and brings decades of research in
architecture to bear, including vector processing and static scheduling, and
combines them with new specialized functional units (\autoref{sec:FUs}) and
scheduling algorithms (\autoref{sec:scheduler}) to design a programmable FHE
accelerator. We implement the main components of F1 in RTL and synthesize them
in a commercial 14nm/12nm process. With a modest area budget of 151\,mm$^2$,
our F1 implementation provides 36 tera-ops/second of 32-bit modular arithmetic,
64\,MB of on-chip storage, and a 1\,TB/s high-bandwidth memory. We evaluate F1
using cycle-accurate simulation running complete FHE applications, and
demonstrate speedups of 1,200$\times$--17,000$\times$ over state-of-the-art
software implementations. These dramatic speedups counter most of FHE's
overheads and enable new applications. For example, F1 executes a deep learning
inference that used to take 20 minutes in 240 milliseconds, enabling secure
real-time deep learning in the cloud.

\section{Motivating CraterLake}

Unfortunately, F1 is efficient only on a limited subset of simple FHE
computations---those of \emph{shallow multiplicative depth}. For example, F1
can run neural network inference efficiently only for networks with few layers
(3-6), but it cannot accelerate state-of-the-art deep neural networks (DNNs)
with tens to hundreds of layers.

This limitation stems from the characteristics of FHE schemes: each ciphertext
has some associated noise, which grows with each homomorphic operation, and
especially with multiplications. If noise becomes too large, it garbles the
message, making decryption impossible. Larger ciphertexts tolerate more noise
before becoming undecryptable. However, operations on larger ciphertexts are
also more expensive. To enable computations of unbounded depth, ciphertexts can
be ``refreshed'' using a procedure called \emph{bootstrapping} that reduces
noise. But bootstrapping is expensive, so ciphertexts must be very large (10s
of MBs) for bootstrapping to be efficient.

FHE accelerators prior to CraterLake do not efficiently handle unbounded-depth computations
because they natively support vectors of a limited size and they use algorithms
that scale poorly to the large ciphertexts in high-depth programs. As a result,
they can only run small FHE computations, and they do not support sufficient
depth to run the full bootstrapping procedure.

CraterLake addresses these challenges and is thus the first FHE accelerator
to support \emph{FHE computations of unbounded depth}. To achieve this, CraterLake
contributes new algorithms, specialized functional units, hardware architecture,
and compiler techniques that overcome the key challenge of deep FHE
computations---its extreme data movement demands.

\paragraph{Deep FHE is limited by data movement:}
While even shallow FHE programs require large operands and can
easily become memory bandwidth bound (\autoref{sec:firstissues}),
this issues is even more pronounced for deep FHE programs.
Concretely, supporting unbounded-depth computations requires vectors of 64K
elements with 1,600 bits per element. This takes 25\,MB per ciphertext,
12$\times$ larger than what prior FHE accelerators target.

Moreover, prior work has employed FHE algorithms that require even larger
amounts of auxiliary data. For example, multiplying 2\,MB ciphertexts in F1
requires 32\,MB of auxiliary data, and scaling their algorithm to 26\,MB
ciphertexts would require over 1.3\,GB of auxiliary data---far too large to fit
on-chip. To tackle this challenge, CraterLake's \emph{key insight} is to adopt
an FHE algorithm called \emph{boosted keyswitching}
(\autoref{sec:keyswitching}) that eliminates most of the auxiliary data,
reducing this overhead from 1.3\,GB to 52.5\,MB. Boosted keyswitching also
reduces computation costs. However, this new algorithm is a poor match for
prior accelerators, including F1: it is dominated by simple operations where
these designs have limited efficiency, and makes poor use of the specialized
functional units that prior designs leverage.

Beyond being inefficient, F1 suffers from a hard-to-scale vector multicore
architecture: to support the needed non-SIMD operations with reasonable cost,
it implement multiple independent cores with narrower vector datapaths.
However, this causes excessive inter-core communication, and the high-bandwidth
interconnect needed grows superlinearly with the number of cores.

\paragraph{Deep FHE demands new hardware techniques:}
To tackle these challenges, we introduce the \emph{CraterLake} architecture
(\autoref{sec:overview}, \autoref{sec:architecture}), the first FHE accelerator
that achieves high performance on unbounded FHE programs. CraterLake is a
wide-vector \emph{uniprocessor} with specialized functional units. Like F1, the
design is statically scheduled to leverage the regularity of FHE computations.
We contribute several new techniques that make this possible, including:
\begin{compactitem}
\item A new extremely wide (2,048 lanes) vector uniprocessor architecture that
    spreads each vector operation across the chip, departing from F1's
    multicore architectures. The uniprocessor approach reduces the number of
    concurrent operations, which minimizes footprint, reducing off-chip
    traffic, and simplifies the compiler.
\item An efficient implementation of the above architecture, which is
    challenging for non-SIMD FHE operations, NTTs and automorphisms, by
    decomposing these operations in a novel way that allows the use of a
    \emph{fixed transpose network} among physically distributed groups of
    lanes. This reduces on-chip data movement and interconnect cost over F1's
    approach.
\item A new functional unit that encapsulates the bulk of operations in boosted
    keyswitching, improving efficiency and enabling high utilization across
    ciphertexts of all sizes.
\item A new functional unit that generates half of the required auxiliary data
    on the fly (reducing overheads from 52\,MB to 26\,MB), saving on-chip
    storage and memory bandwidth.
\item A vector chaining technique that builds long FU pipelines to enable many
    concurrent operations with few register ports.
\end{compactitem}

To program CraterLake, we develop a novel compiler (\autoref{sec:compiler}) that
produces efficient code from high-level FHE programs. The compiler schedules
operations to maximize reuse, decouples data movement from computation, and
adapts the state-of-the-art bootstrapping algorithm to achieve high
utilization~\cite{bossuat:crypto21:efficient}.

We evaluate CraterLake through a combination of simulation and RTL synthesis
(to find its area and power). We use a broad range of FHE benchmarks, including
programs with high multiplicative depth that require bootstrapping. CraterLake
outperforms a scaled-up and improved version of F1, by gmean 11.2$\times$ on
these deep computations, and is 4,600$\times$ faster than a 32-core CPU. These
speedups enable new use cases for FHE. For example, deep neural networks like
ResNet take 23 minutes per inference on the CPU, whereas \name achieves 250
\emph{milliseconds} per inference, enabling real-time private deep learning.
