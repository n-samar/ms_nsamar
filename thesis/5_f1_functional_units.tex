\section{Functional Units}
\label{sec:FUs}

In this section, we describe F1's novel functional units.
These include the first vectorized automorphism unit (\autoref{sec:automorphism}),
the first fully-pipelined flexible NTT unit (\autoref{sec:fourStepNTT}),
and a new simplified modular multiplier adapted to FHE (\autoref{sec:modMult}).

\subsection{Automorphism unit}\label{sec:automorphism}

\figFOneAutomorphism

Because F1 uses $E$ vector lanes, each residue polynomial
is stored and processed as $G$ groups, or \emph{chunks}, of $E$ elements each ($N=G\cdot E$).
An automorphism $\sigma_k$ maps the element at index $i$ to index $ki \textrm{ mod } N$;
there are $N$ automorphisms total, two for each odd $k < N$ (\autoref{sec:fhe_operation}).
The key challenge in designing an automorphism unit is that these permutations are hard to vectorize:
we would like this unit to consume and produce $E=$128 elements/cycle, but the vectors
are much longer, with $N$ up to 16\,K, and elements are permuted across different chunks.
Moreover, we must support variable $N$ \emph{and} all automorphisms.

Standard solutions fail: a 16\,K$\times$16\,K crossbar is much too large;
a scalar approach, like reading elements in sequence from an SRAM, is too slow (taking $N$ cycles);
and using banks of SRAM to increase throughput runs into frequent bank conflicts:
each automorphism ``spreads''~elements with a different stride, so regardless of the banking scheme,
some automorphisms will map many consecutive elements to the~same~bank.

We contribute a new insight that makes vectorizing automorphisms simple:
if we interpret a residue polynomial as a $G \times E$ matrix,
an automorphism can always be decomposed into two independent \emph{column} and \emph{row permutations}.
If we transpose this matrix, both column and row permutations can 
be applied \emph{in chunks of $E$ elements}. \autoref{fig:f1automorphism} shows an example 
of how automorphism $\sigma_3$ is applied to a residue polynomial
with $N=16$ and $E=4$ elements/cycle.
Note how the permute column and row operations are local to each $4$-element chunk.
Other $\sigma_k$ induce different permutations, but with the same row/column structure.

\figFOneautfu
Our automorphism unit, shown in \autoref{fig:f1aut_fu},
uses this insight to be both vectorized (consuming $E=128$ elements/cycle) and fully pipelined.
Given a residue polynomial of $N=G\cdot E$ elements, the automorphism unit first applies the column permutation
to each $E$-element input.
Then, it feeds this to a \emph{transpose unit} that reads in the whole residue polynomial interpreting it as a $G\times E$ matrix,
and produces its transpose $E\times G$.
The transpose unit outputs $E$ elements per cycle (outputting multiple rows per cycle when $G < E$).
Row permutations are applied to each $E$-element chunk, and the reverse transpose is applied.


Further, we decompose both the row and column permutations into a pipeline of sub-permutations that are \textit{fixed in hardware},
with each sub-permutation either applied or bypassed based on simple control logic; this avoids using crossbars for the $E$-element permute row and column operations.

\figFOneQuadrantSwap

\paragraph{Transpose unit:}
Our \textit{quadrant-swap transpose} unit transposes an $E \times E$ (e.g., $128\times 128$) matrix by recursively decomposing it into quadrants and exploiting
the identity

%\vspace{-12pt}
\begin{equation*}
  \left[ \begin{array}{c|c}
      \texttt{A} & \texttt{B}\\
      \hline
      \texttt{C} & \texttt{D}
  \end{array}\right]^{\textrm{T}} =   \left[ \begin{array}{c|c}
      \texttt{A}^{\textrm{T}} & \texttt{C}^{\textrm{T}} \\
      \hline
      \texttt{B}^{\textrm{T}} & \texttt{D}^{\textrm{T}}
  \end{array}\right].
\end{equation*}

The basic building block is a $K \times K$ \textit{quadrant-swap} unit, which swaps quadrants \texttt{B} and \texttt{C}, as shown in \autoref{fig:f1quadrantSwap}(left). Operationally, the quadrant swap procedure consists of three steps, each taking $K/2$ cycles:
  \emph{(1)} Cycle \texttt{i} in the first step reads \texttt{A[i]} and \texttt{C[i]} and stores them in \texttt{top[i]} and \texttt{bottom[i]}, respectively.
\emph{(2)} Cycle \texttt{i} in the second step reads \texttt{B[i]} and \texttt{D[i]}. The unit activates the first swap MUX and the bypass line, thus storing \texttt{D[i]} in \texttt{top[i]} and outputing \texttt{A[i]} (by reading from \texttt{top[i]}) and \texttt{B[i]} via the bypass line.
\emph{(3)} Cycle \texttt{i} in the third step outputs \texttt{D[i]} and \texttt{C[i]} by reading from \texttt{top[i]} and \texttt{bottom[i]}, respectively. The second swap MUX is activated so that \texttt{C[i]} is on top.

Note that step $3$ for one input can be done in parallel with step $1$ for the next, so the unit is \emph{fully pipelined}.

The transpose is implemented by a full $E \times E$ quadrant-swap followed by $\log_2E$ layers of smaller transpose units
to recursively transpose \texttt{A}, \texttt{B}, \texttt{C}, and \texttt{D}. \autoref{fig:f1quadrantSwap} (right) shows an implementation for $E=8$. Finally, by selectively bypassing some of the initial quadrant swaps,
this transpose unit also works for all values of $N$ ($N=G\times E$ with power-of-2 $G < E$).

Prior work has implemented transpose units for signal-processing applications,
either using registers~\cite{wang2018pipelined,zhang2020novel} or with custom SRAM designs~\cite{shang2014single}.
Our design has three advantages over prior work: it uses standard SRAM memory,
so it is dense without requiring complex custom SRAMs;
it is fully pipelined; and it works for a wide range of dimensions.

\subsection{Four-step NTT unit}\label{sec:fourStepNTT}



%The Cooley-Tukey algorithm \cite{cooley:moc65:algorithm} can be parallelized in hardware into a so-called \textit{NTT butterfly} pipeline.
%However, this pipeline requires the datapath to be as wide as the number of elements we want
%to apply the NTT to. This means we would need a $16K$-element datapath to support $N=16K$, which is too wide.
%Ideally, instead of unrolling the datapath prior to an NTT, we would fix it to $128$-element hardware vectors,
%with each ciphertext stored in $8$-$128$ hardware vectors (depending on the value of $N$).

There are many ways to implement NTTs in hardware:
an NTT is like an FFT~\cite{cooley:moc65:algorithm}
but with a butterfly that uses modular multipliers.
We implement $N$-element NTTs (from 1K to 16K) as a composition
of smaller $E$=128-element NTTs,
since implementing a full 16K-element NTT datapath is prohibitive.
The challenge is that standard approaches result in memory access patterns
that are hard to vectorize.

\figFOneFourStepNTT

To that end, we use the \textit{four-step variant} of the FFT algorithm~\cite{bailey:supercomputing89:FFTs},
which adds an extra multiplication to produce a vector-friendly decomposition.
\autoref{fig:f1fourStepNTT} illustrates 
%a simple % dsm: "simple" :)
our
four-step NTT pipeline for $E=4$;
we use the same structure with $E=128$.
The unit is fully pipelined and consumes $E$ elements per cycle.
To compute an $N=E\times E$ NTT, the unit first computes an $E$-point NTT on each $E$-element group,
multiplies each group with twiddles,
transposes the $E$ groups, and computes another $E$-element NTT on each transpose.
The same NTT unit implements the inverse NTT
by storing multiplicative factors (\textit{twiddles}) required for both forward and inverse NTTs in a small \textit{twiddle SRAM}.

Crucially, we are able to support all values of $N$ using a single four-step NTT pipeline by conditionally bypassing layers in the second NTT butterfly.
We use the same transpose unit implementation as with automorphisms.

Our four-step pipeline supports negacyclic NTTs (NCNs), which are more efficient than standard non-negacyclic NTTs (that would require padding, \autoref{sec:algoInsights}). Specifically, we extend prior work~\cite{poppelmann2015high,roy2014compact,lyubashevsky:tact10:ideal} in order to support \emph{both} forward and inverse NCNs using the same hardware as for the standard NTT. Namely, prior work shows how to either \emph{(1)} perform a forward NCN via a standard decimation-in-time (DIT) NTT pipeline, or \emph{(2)} perform an inverse NCN via a standard decimation-in-frequency (DIF) NTT pipeline. The DIF and DIT NTT variants use different hardware; therefore, this approach requires separate pipelines for forward and inverse NCNs. Prior work~\cite{lyubashevsky:tact10:ideal} has shown that separate pipelines can be avoided by adding a multiplier either before or after the NTT: doing an \emph{inverse} NCN using a \emph{DIT} NTT requires a multiplier unit \emph{after} the NTT, while doing a \emph{forward} NCN using a \emph{DIF} NTT requires a multiplier unit \emph{before} the NTT.

We now show that \emph{both} the forward and inverse NCN can be done in the same standard four-step NTT pipeline, with \emph{no additional hardware}. This is because the four-step NTT already has a multiplier and two NTTs in its pipeline. We set the first NTT to be decimation-in-time and the second to be decimation-in-frequency (\autoref{fig:f1fourStepNTT}). To do a forward NTT, we use the forward NCN implementation via DIT NTT for the first NTT; we modify the contents of the Twiddle SRAM so that the multiplier does the pre-multiplication necessary to implement a forward NCN in the second NTT (which is DIF and thus requires the pre-multiplication). Conversely, to do an inverse NTT, we modify the Twiddle SRAM contents to do the post\hyp{}mul\-ti\-pli\-ca\-tion necessary to implement an inverse NCN in the first NTT (which is DIT); and we use the inverse NCN imple\-men\-ta\-tion via DIF NTT for the second NTT.

The NTT unit is large: each of the 128-element NTTs requires $E(\log (E)-1)/2$=384 multipliers,
and the full unit uses 896 multipliers.
But its high throughput improves performance over many low-throughput NTTs (\autoref{sec:evaluation}). %by gmean 2.5$\times$ on relevant benchmarks (\autoref{tbl:f1sensitivity}); this is because using many small NTT units in parallel causes the memory footprint to quickly exceed scratchpad capacity.
This is the first implementation of a fully-pipelined four-step NTT unit, 
improving NTT performance by 1,600$\times$ over the state of the art (\autoref{sec:perf}).


\subsection{Optimized modular multiplier}\label{sec:modMult}
\tblFOneModMult

Modular multiplication computes $a\cdot b \textrm{ mod } q$.
This is the most expensive and frequent operation.
Therefore, improvements to the modular multiplier have an almost
linear impact on the computational capabilities of an FHE accelerator.

Prior work~\cite{mert:euromicro19:design}
recognized that a Montgomery multiplier~\cite{montgomery:mom85:modular} within NTTs can be improved by leveraging
the fact that the possible values of modulus $q$ are restricted by the number of elements the NTT is applied to.
We notice that if we only select moduli $q_i$, such that $q_i = -1 \textrm{ mod } 2^{16}$,
we can remove a mutliplier stage from~\cite{mert:euromicro19:design};
this reduces area by 19\% and power by 30\% (\autoref{tbl:f1modMult}).
The additional restriction on $q$ is acceptable because FHE
requires at most 10s of moduli~\cite{gentry:crypto2012:homomorphic},
and our approach allows for 6,186~prime~moduli.
