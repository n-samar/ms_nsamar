\section{Conclusion}\label{sec:conclusion}

For widespread adoption of FHE, accelerators must efficiently support deep computations.
\name is the first accelerator to achieve this.
By adopting state-of-the-art algorithms and using them to design \name,
we target a new regime of FHE not explored by prior approaches.
Through new architectural and compiler techniques, \name addresses the overheads
of deep computations and provides order-of-magnitude speedups over prior accelerators
(even when that prior work is scaled up and allowed several idealizations).
As a result, \name enables new applications for FHE, such as real-time inference
using deep neural networks like ResNet or LSTMs.


%% overheads inherent in deep computations. Prior work falls short of this goal 
%% because FHE is well- suited to vector processors and does not require any dynamic 
%% flow control; this makes CPUs inadequate for FHE. GPUs are also inappropriate 
%% because (1) although many FHE operations are amenable to GPUs’ vector processor 
%% style, some crucial operations have complicated dependencies between vectors 
%% which cannot be handled well by GPUs, and (2) FHE relies on modular arithmetic 
%% and has little use for GPUs’ floating-point pipelines. \name accelerates 
%% computation on encrypted data with arbitrary depth. Our main contributions are:
%% 1) First accelerator to target arbitrary depth computations and show that it can be done.
%% 2) Show that new keyswitching algorithm is actually the only way to go if you 
%% want deep computation (for some reason, prior work like heax, bajard, F1 does not use it)
%% 3) Builds customized pipelines to accelerate critical computation. Argues why 
%% customized pipelines are the way to go and what they should be.
%% 4) Providing a simple interface to the programmer. There
%% is sufficient parallelism within FHE operations so programmer does not have 
%% to worry about finding parallelism themselves. This increases utilization.
%% 5) Improve on FHE algorithms for deep computations cuz nobody else saw the use 
%% of these algorithms (considered intractable)
%% 6) Choosing the optimal keyswitching algorithm for CKKS.
