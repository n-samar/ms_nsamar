\section{Experimental Methodology}\label{sec:methodology}

\paragraph{Modeled system:}
We evaluate our CraterLake implementation from \autoref{sec:implementation}
using a cycle-accurate simulator to execute CraterLake programs. We use
activity-level energies from synthesized components to produce energy
breakdowns.

\paragraph{Benchmarks:}
We use several FHE programs to evaluate CraterLake. All programs come from
state-of-the-art software implementations, and use the CKKS scheme. To show
that CraterLake is efficient on unbounded computations, we use four \emph{deep}
benchmarks, which have a high multiplicative depth and require bootstrapping.
We also use four \emph{shallow} benchmarks, with low multiplicative depth and
no bootstrapping, to show that CraterLake is also efficient at low depths.

We meet 80-bit security for all benchmarks by using a combination of 2-digit
and 1-digit keyswitching (\autoref{sec:boostedSecurity}). We also use
non-sparse keys and the most recent bootstrapping
techniques~\cite{bossuat2021efficient} in order to maximize our multiplicative
budget without losing precision. We later benchmark performance for 128-bit
security and beyond (\autoref{sec:moreSecurity}). We use the LWE
estimator~\cite{albrecht2018estimate} to derive security parameters.

\paragraph{Deep benchmarks} include:

\noindent \emph{\textbf{(1) LSTM}} is a Long-Term Short-Term (LSTM) NLP
benchmark~\cite{podschwadt:2020:classification}. This benchmark boils down to
computing $h_{i+1} = \sigma(W_0h_i + W_1x_i)$ many times. $\sigma$ is an
activation function approximated by a degree-3 polynomial, and $W_0h_i$,
$W_1x_i$ are 128$\times$128 matrix-vector multiplies. This computation is
multiplicatively deep and requires 50 bootstrappings per inference.

\noindent \emph{\textbf{(2) ResNet-20}} is an FHE
implementation~\cite{lee:2021:privacy} of the ResNet-20 DNN. We benchmark an
inference on a single~encrypted~image.

\tblBenchmarksAndPerformanceBreakdown

\noindent \emph{\textbf{(3) Logistic regression}}
uses the HELR algorithm~\cite{han:aaai19:logistic}, which is based on CKKS. We
compute many batches of logistic regression training with 256 features, and 256
samples per batch, starting at computational depth $L$=38. This benchmark is
different from the one reported in F1, as it performs multiple logistic
regression iterations. F1 reported performance on only a single iteration,
thereby avoiding frequent bootstrapping that is necessary for running multiple
training iterations.

\noindent \emph{\textbf{(4) Fully-packed bootstrapping}}
takes an $L$=3 and $N$=64K ciphertext with an exhausted multiplicative budget
and refreshes it by bringing it up to $L$=57, then performs the bootstrapping
computation to obtain a usable ciphertext at a lower budget. The
\emph{fully-packed} version implies the ciphertext uses all $N$/2=32K available
slots. Bootstrapping costs grow with the number of slots (both in
multiplicative depth and compute).

We use the state-of-the-art fully packed bootstrapping
algorithm~\cite{mouchet2020lattigo}, and use Lattigo's
implementation~\cite{lattigo} as the baseline. We tune CraterLake's
bootstrapping implementation to maximize performance as discussed in
\autoref{sec:algorithmicInsights}.

For consistency, \emph{we also use this bootstrapping algorithm in all
benchmarks that require bootstrapping}. This is important, because this
algorithm is not yet widely implemented in other libraries, so the original
ResNet-20 and LogReg used much slower bootstrapping algorithms. In fact, the
cost of older bootstrapping algorithms grows very quickly with the number of
plaintext elements encoded in each ciphertext, so the baselines used
\emph{partially packed ciphertexts} (e.g., packing 128 elements per $N$=64K
ciphertext) to reduce overall overheads. But with efficient bootstrapping,
using fully packed ciphertexts is more efficient. For instance, we modify
ResNet-20 to pack all channels into a single ciphertext before bootstrapping.
This reduces the number of bootstrappings by 38$\times$ and improves
performance on all hardware platforms by about 10$\times$.

\paragraph{Shallow benchmarks} match those used for F1
(\autoref{sec:f1_methodology}). These benchmarks do not use bootstrapping and
their max $L$ is between 4 and 8.

Finally, we also benchmark \emph{unpacked bootstrapping}, which bootstraps a
ciphertext that packs a single element. This makes it shallower ($L{\leq}23$)
and less computationally demanding, but performance per slot is a lot worse
than fully packed bootstrapping. Thus, it isn't used much in practice. We
include it because it is the bootstrapping benchmark used in
F1.

\paragraph{Compared systems:}
We compare CraterLake with a CPU system. We use a 32-core, 64-thread, 3.5\,GHz
AMD Ryzen Threadripper PRO 3975WX; at 420mm$^2$ in a mix of 7nm and 12nm
technology, this CPU has a comparable transistor count and power budget (280\,W
TDP) to CraterLake. This system is significantly more powerful than the F1
baseline (\autoref{sec:f1_methodology}) as we chose the baselines to match the
area and power budgets of the respective accelerators.

We also compare performance to prior accelerators, in particular to
F1. For fairness, we evaluate \emph{F1+}, a version
of F1 that is scaled to a 256\,MB 32-bank scratchpad, 32 compute clusters with
256 lanes each, and 1\,MB register file per cluster. This makes F1+ have the
same or higher throughput on basic operations as CraterLake. However, F1+ takes
636\,mm$^2$, 35\% more than CraterLake, because its network scales poorly:
F1+'s on-chip network takes 160\,mm$^2$, 16$\times$ more than CraterLake's
fixed permutation network. This large overhead shows that CraterLake's novel
hardware organization is crucial to scale.

Finally, although F1 is tailored to standard keyswitching, boosted keyswitching
becomes more efficient for $L>14$. Thus, F1+ uses the most efficient
keyswitching algorithm at each level. In short, \emph{these changes allow
comparing the F1 and CraterLake architectures} without the confounding factors of
different hardware budgets or subpar algorithms.

