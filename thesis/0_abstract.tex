Fully Homomorphic Encryption (FHE) enables offloading computation to untrusted
servers with cryptographic privacy. Despite its attractive security, FHE is not
yet widely~adopted due to its prohibitive overheads, about 10,000$\times$ over
unencrypted computation.

Hardware acceleration is an attractive approach to bridging this performance
gap, but it brings new challenges and opportunities, chiefly: complex
operations on long vectors, statically-scheduled computation, and the data
movement bottleneck. This thesis presents two FHE accelerators that address
these challenges: F1 and CraterLake.

F1 is the first \emph{programmable} FHE accelerator, i.e., capable of executing
full FHE programs. F1 is a wide-vector processor with novel functional units
deeply specilized to FHE primitives. It speeds up shallow FHE computations
(i.e., those of limited multiplicative depth) by gmean 5,400$\times$ over the
state-of-the-art. Unfortunately, F1 becomes memory bandwidth bound on deeper
computations (e.g., deep neural networks). This is because deep FHE programs
require very large ciphertexts (tens of MBs each) and different algorithms that
F1 does not support well.

CraterLake addresses these shortcomings and is the first accelerator to
effectively speed up FHE programs of unbounded depth.  CraterLake introduces a
new hardware architecture that efficiently scales to very~large ciphertexts,
novel functional units to accelerate key kernels, and new algorithms and
compiler techniques to reduce data movement. These advances help CraterLake
deliver 11.2$\times$ the performance of F1 on deep benchmarks, even when we
scale F1's architecture to the size of CraterLake.

Critically both F1 and CraterLake are fully \emph{programmable}, i.e., capable
of executing \emph{any} FHE programs. Additionally, we show that CraterLake
delivers similar speedups over CPU on a diverse set of benchmarks.
