Fully Homomorphic Encryption (FHE) enables offloading computation to untrusted
servers with cryptographic privacy. Despite its attractive security, FHE is not
yet widely~adopted due to its prohibitive overheads, about 10,000$\times$ over
unencrypted computation.

Hardware acceleration is an attractive approach to bridging this performance
gap, but this brings new challenges. This thesis presents addresses these
challenges and presents two FHE accelerators: F1 and CraterLake.

F1 speeds up shallow computations (i.e., shallow multiplicative depth) by gmean
5,400$\times$ over the state-of-the-art. Unfortunately, F1 becomes memory
bandwidth bound on deeper computations, like deep neural networks. CraterLake
is the first accelerator to effectively speed up unbounded depth FHE programs.
It builds on and addresses the shortcomings of F1. Specifically, deep FHE
programs require very large ciphertexts (tens of MBs each) and different
algorithms that prior work does not support well. To tackle this challenge,
\name introduces a new hardware architecture that efficiently scales to
very~large ciphertexts, novel functional units to accelerate key kernels, and
new algorithms and compiler techniques to reduce data movement. These advances
help CraterLake deliver 11.2$\times$ the performance of F1, even when we scale
F1's architecture to the size of CraterLake.

Critically both F1 and CraterLake are fully \emph{programmable}, i.e., capable
of executing full FHE programs, and bring similar speedups across a diverse set
of benchmarks.
