\chapter{Additional Related Work}\label{sec:related}

We now present related work not discussed so far.

\paragraph{FHE accelerators:}
Prior work has proposed accelerators for individual FHE operations, but not
full FHE
computations~\cite{cousins:hpec12:sipher-fpga,cousins:hpec14:fpga-he,cousins:tetc17:fpga-he,doroz:tc15:accelerating-fhe,roy:hpca19:fpga-he,mert:tvlsi20:bfv-accel,migliore:tecs17:he-karatsuba,riazi:asplos20:heax,turan:tc20:heaws}.
These designs target FPGAs and rely on a host processor;
\autoref{sec:fhe_analysis} discussed their limitations. Early designs
accelerated small primitives like NTTs, and were dominated by host-FPGA
communication. State-of-the-art accelerators execute a full homomorphic
multiplication independently: Roy et al.~\cite{roy:hpca19:fpga-he} accelerate
B/FV multiplication by 13$\times$ over a CPU; HEAWS~\cite{turan:tc20:heaws}
accelerates B/FV multiplication, and uses it to speed a simple benchmark by
5$\times$; and HEAX~\cite{riazi:asplos20:heax} accelerates CKKS multiplication
and keyswitching by up to 200$\times$. These designs suffer high data movement
(e.g., HEAX does not reuse keyswitch hints) and use fixed pipelines with
relatively low-throughput FUs.

We have shown that accelerating FHE programs requires a different approach:
data movement becomes the key constraint, requiring new techniques to extract
reuse {across} homomorphic operations; and fixed pipelines cannot support the
operations of even a single benchmark. Instead, F1 and CraterLake achieves
flexibility and high performance by exploiting wide-vector execution with
high-throughput FUs. This lets F1 and CraterLake execute not only full
applications, but different FHE schemes.


\paragraph{HE-MPC accelerators are hampered by communication:}
To avoid the overheads of FHE, recent work has proposed accelerators for
private deep learning that combine shallow homomorphic encryption (HE) with
multi-party computation (MPC): Gazelle~\cite{juvekar2018gazelle} and
Cheetah~\cite{reagen:hpca21:cheetah}. These systems require very frequent
communication with the client, essentially after every single level of
multiplication. So while they reduce accelerator overheads, they are limited by
high client-server communication and client encryption/decryption overheads.
Delphi~\cite{mishra2020delphi} shows that each DNN inference takes gigabytes of
traffic, which dominates performance; yet, the above accelerators do not
consider this traffic. CHOCO~\cite{vanderhagen:arxiv21:choco} shows that, even
after accelerating client operations, communication costs dominate.

By dramatically accelerating bootstrapping, CraterLake eliminates the high
communication costs of HE-MPC and shallow HE designs. To avoid bootstrapping,
these prior approaches require the \emph{client} to receive, reencrypt, and
resend ciphertexts that have exhausted their multiplicative budgets. In our
benchmarks, avoiding each bootstrapping would require transferring over 13\,MB
between client and server (as the client must resend the ciphertext with a
reasonable noise budget). Even if we ignore client computation and network
latency, on a 100\,Mbps connection this would require over 1 \emph{second} per
ciphertext. By contrast, CraterLake bootstraps this ciphertext in 3.9\,ms,
256$\times$ faster. Bootstrapping also allows the client to send narrow inputs
(e.g., with 32 instead of 1,500 bits per coefficient), which the server can
bootstrap before computation. This greatly reduces encryption and network
overheads.

\paragraph{GPUs are inefficient on FHE:}
Prior work has studied the use of GPUs to accelerate
FHE~\cite{wang:hpec12:fhe-gpu,wang:tc13:fhe-gpu,wang:iscas14:leveled-gpu,al:emerging19:implementation,jung2021over}.
While the data-parallel nature of GPUs may seem a good fit for FHE, these
efforts achieve modest speedups over multicore CPUs. This is because GPUs lack
modular arithmetic, cannot implement all-to-all operations like NTTs and
automorphisms efficiently, and their on-chip memories are too small to enable
sufficient reuse (\autoref{fig:gmeanVsStorage}), despite their use of HBM.
Specifically, state-of-the-art GPU approaches carefully tune algorithms to
achieve high off- and on-chip bandwidth utilization~\cite{jung2021over};
however, \cite{jung2021over} is 200$\times$ slower than CraterLake. This shows that
CraterLake's high reuse is crucial: to achieve the same throughput as CraterLake, a GPU
would need over 100\,TB/s of memory bandwidth.
