% -*-latex-*-
% 
% For questions, comments, concerns or complaints:
% thesis@mit.edu
% 
%
% $Log: cover.tex,v $
% Revision 1.8  2008/05/13 15:02:15  jdreed
% Degree month is June, not May.  Added note about prevdegrees.
% Arthur Smith's title updated
%
% Revision 1.7  2001/02/08 18:53:16  boojum
% changed some \newpages to \cleardoublepages
%
% Revision 1.6  1999/10/21 14:49:31  boojum
% changed comment referring to documentstyle
%
% Revision 1.5  1999/10/21 14:39:04  boojum
% *** empty log message ***
%
% Revision 1.4  1997/04/18  17:54:10  othomas
% added page numbers on abstract and cover, and made 1 abstract
% page the default rather than 2.  (anne hunter tells me this
% is the new institute standard.)
%
% Revision 1.4  1997/04/18  17:54:10  othomas
% added page numbers on abstract and cover, and made 1 abstract
% page the default rather than 2.  (anne hunter tells me this
% is the new institute standard.)
%
% Revision 1.3  93/05/17  17:06:29  starflt
% Added acknowledgements section (suggested by tompalka)
% 
% Revision 1.2  92/04/22  13:13:13  epeisach
% Fixes for 1991 course 6 requirements
% Phrase "and to grant others the right to do so" has been added to 
% permission clause
% Second copy of abstract is not counted as separate pages so numbering works
% out
% 
% Revision 1.1  92/04/22  13:08:20  epeisach

% NOTE:
% These templates make an effort to conform to the MIT Thesis specifications,
% however the specifications can change.  We recommend that you verify the
% layout of your title page with your thesis advisor and/or the MIT 
% Libraries before printing your final copy.
\title{Designing a Programmable Hardware Accelerator for Fully Homomorphic Encryption}

\author{Axel S. Feldmann}
% If you wish to list your previous degrees on the cover page, use the 
% previous degrees command:
%       \prevdegrees{A.A., Harvard University (1985)}
% You can use the \\ command to list multiple previous degrees
%       \prevdegrees{B.S., University of California (1978) \\
%                    S.M., Massachusetts Institute of Technology (1981)}
\prevdegrees{B.S. in Computer Science\linebreak
Carnegie Mellon University, 2019}
\department{Electrical Engineering and Computer Science}

% If the thesis is for two degrees simultaneously, list them both
% separated by \and like this:
% \degree{Doctor of Philosophy \and Master of Science}
\degree{Master of Science in Electrical Engineering and Computer Science}

% As of the 2007-08 academic year, valid degree months are September, 
% February, or June.  The default is June.
\degreemonth{June}
\degreeyear{2021}
\thesisdate{May 20, 2021}

%% By default, the thesis will be copyrighted to MIT.  If you need to copyright
%% the thesis to yourself, just specify the `vi' documentclass option.  If for
%% some reason you want to exactly specify the copyright notice text, you can
%% use the \copyrightnoticetext command.  
%\copyrightnoticetext{\copyright IBM, 1990.  Do not open till Xmas.}

% If there is more than one supervisor, use the \supervisor command
% once for each.
\supervisor{Daniel Sanchez}{Associate Professor of Electrical Engineering and Computer Science}

% This is the department committee chairman, not the thesis committee
% chairman.  You should replace this with your Department's Committee
% Chairman.
\chairman{Leslie A. Kolodziejski}{
Professor of Electrical Engineering and Computer Science\\
Chair, Department Committee on Graduate Students}

% Make the titlepage based on the above information.  If you need
% something special and can't use the standard form, you can specify
% the exact text of the titlepage yourself.  Put it in a titlepage
% environment and leave blank lines where you want vertical space.
% The spaces will be adjusted to fill the entire page.  The dotted
% lines for the signatures are made with the \signature command.
\maketitle

% The abstractpage environment sets up everything on the page except
% the text itself.  The title and other header material are put at the
% top of the page, and the supervisors are listed at the bottom.  A
% new page is begun both before and after.  Of course, an abstract may
% be more than one page itself.  If you need more control over the
% format of the page, you can use the abstract environment, which puts
% the word "Abstract" at the beginning and single spaces its text.

%% You can either \input (*not* \include) your abstract file, or you can put
%% the text of the abstract directly between the \begin{abstractpage} and
%% \end{abstractpage} commands.

% First copy: start a new page, and save the page number.
\cleardoublepage
% Uncomment the next line if you do NOT want a page number on your
% abstract and acknowledgments pages.
% \pagestyle{empty}
\setcounter{savepage}{\thepage}
\begin{abstractpage}

\emph{``This innovation that this industry talks about so much is bullshit. Anybody
can innovate. Don't do this `think different'; don't do this big `innovation'
thing. Screw that. It's meaningless. 99\% of it is: get the work done.''}

--- Linus Torvalds
\vspace{1cm}

Fully Homomorphic Encryption (FHE) allows computing on encrypted data, enabling secure offloading of computation to untrusted servers.
Though it provides ideal security, FHE is expensive when executed in software, 4 to 5 orders of magnitude slower than computing on unencrypted data.
These overheads are a major barrier to FHE's widespread adoption.

We present \name, the first FHE accelerator that is programmable, i.e., capable of executing full FHE programs.
\name builds on an in-depth architectural analysis of the characteristics of FHE computations
that reveals acceleration opportunities. \name is a wide-vector processor with novel functional units deeply specialized to FHE primitives, 
such as modular arithmetic, number-theoretic transforms, and structured permutations.

% axelf: tentative?
Due to the static nature of FHE computations, \name uses an exposed ISA, requiring novel compilation techniques to statically schedule all compute and data movement. We design a compiler that efficiently maps FHE programs onto \name hardware and maximizes reuse of on-chip data, helping to reduce data movement bottlenecks. 
The compiler leverages \name's explicitly managed scratchpad to decouple computation from data movement, a necessary ingredient in achieving high performance given the large size of FHE operands.

% This organization provides so much compute throughput that data movement becomes the key bottleneck.
% Thus, \name is primarily designed to minimize 
% data movement.
% Hardware provides an explicitly-managed memory hierarchy and mechanisms to decouple data movement from execution.
% A novel compiler leverages these mechanisms to maximize reuse and schedule off-chip and on-chip data movement.
 
We evaluate \name using cycle-accurate simulation and RTL synthesis.
\name is the first system to accelerate complete FHE programs,
and outperforms state-of-the-art software implementations by gmean 6,500$\times$ and by up to 17,000$\times$.
These speedups counter most of FHE's overheads and enable new applications, like real-time private deep learning in the cloud.

\end{abstractpage}

% Additional copy: start a new page, and reset the page number.  This way,
% the second copy of the abstract is not counted as separate pages.
% Uncomment the next 6 lines if you need two copies of the abstract
% page.
% \setcounter{page}{\thesavepage}
% \begin{abstractpage}
% % $Log: abstract.tex,v $
% Revision 1.1  93/05/14  14:56:25  starflt
% Initial revision
% 
% Revision 1.1  90/05/04  10:41:01  lwvanels
% Initial revision
% 
%
%% The text of your abstract and nothing else (other than comments) goes here.
%% It will be single-spaced and the rest of the text that is supposed to go on
%% the abstract page will be generated by the abstractpage environment.  This
%% file should be \input (not \include 'd) from cover.tex.



% \end{abstractpage}

\cleardoublepage

\section*{Acknowledgments}

Thank you to all the people that created the environment in which success is
easy. Your work matters and changes lives.

This work was conducted in collaboration with Axel Feldmann, Aleksandar
Krastev, Nicholas Genise, Prof. Srini Devadas, Karim Eldefrawy, Nathan Manohar,
Prof. Ron Dreslinski, Prof. Chris Peikert, and my research advisor Prof. Daniel
Sanchez. Much of this thesis is adapted from joinly written papers. This work
would not have been possible without all of their contributions.


%%%%%%%%%%%%%%%%%%%%%%%%%%%%%%%%%%%%%%%%%%%%%%%%%%%%%%%%%%%%%%%%%%%%%%
% -*-latex-*-
