\usepackage[normalem]{ulem}

%\usepackage{xparse} % makes build fail on mads, ???
\usepackage{verbatim}
\usepackage{graphicx,afterpage}
\usepackage{amsmath}
\usepackage{amssymb}
%\usepackage{amsthm} % dsm: Incompatible with sig-alternate
\usepackage{amsfonts}
%\usepackage[mathscr]{euscript}
%\usepackage{commath}
\usepackage{calc}
\usepackage{xspace}
\usepackage{multirow, booktabs}
\usepackage{paralist} % FIXME(dsm): compactenum/item incompatible with acmart, but we're still using inparaenum...
\usepackage{flushend}
\usepackage{wrapfig}
\usepackage{enumitem}
%\usepackage{enumerate}
% \usepackage{subcaption}
% \usepackage{subfigure}
%\usepackage{dblfloatfix} % to allow two-column floats at the bottom of the page

% dsm: Per ASPLOS 2018 submission instructions, we can have 9pt captions. Problem is, they look worse and don't really save space.
%\captionsetup[figure]{labelfont={small,bf},textfont={small,bf}}
%\captionsetup[table]{labelfont={small,bf},textfont={small,bf}}
%\captionsetup[subfloat]{labelfont={small},textfont={small}}

\setlength{\skip\footins}{9pt}

\usepackage[sort,nocompress]{cite} % autosort groups of citations
% mcj: the following is part of the MICRO 2019 default main.tex
\usepackage[final]{microtype}
%\usepackage[activate={true,nocompatibility},final,tracking=false,kerning=true,spacing=true]{microtype}

%\usepackage[caption=false]{subfig} % for when sig-alternate + caption messes things up
\usepackage{subfig}
\usepackage[labelfont=bf]{caption}
% \renewcommand\thesubfigure{(\alph{subfigure})}

%For fancy tables
\usepackage{array} % decent table formatting
\usepackage{rotating}
%\usepackage[usenames,dvipsnames,svgnames,table]{xcolor}

%% http://tex.stackexchange.com/questions/40283/wrapping-table-column-headings-in-turn-environment
\usepackage{varwidth}
%% \newcommand{\turny}[3][10em]{% \turn[<width>]{<angle>}{<stuff>}
%%   \rlap{\rotatebox{#2}{\begin{varwidth}[t]{#1}#3\end{varwidth}}}%
%% }
\newcommand{\turny}[3][10em]{% \turn[<width>]{<angle>}{<stuff>}
\begin{turn}{#2}
  \begin{varwidth}[t]{#1}
    #3
  \end{varwidth}
\end{turn}
}

%Use this instead of a hyphen to allow the word itself to be hyphenated
\usepackage{hyphenat}
\usepackage[hyphens]{url}

\usepackage{tikz}

\usepackage{listings}

\makeatletter
\let\old@lstKV@SwitchCases\lstKV@SwitchCases
\def\lstKV@SwitchCases#1#2#3{}
\makeatother
\usepackage{lstlinebgrd}
\makeatletter
\let\lstKV@SwitchCases\old@lstKV@SwitchCases

\makeatother

\lstset{
escapeinside={{(@}{@)}},
}

\lstdefinestyle{custompseudocode}{
 aboveskip=0in,
 belowskip=0in,
 abovecaptionskip=0in,
 belowcaptionskip=0in,
 %breaklines=true,
 captionpos=b,
 xleftmargin=\parindent,
 language={},
 morekeywords={define,if,for,while,do},
 showstringspaces=false,
 % dsm: semibold
 basicstyle={\linespread{0.6}\fontseries{sb}\small\ttfamily},
 %basicstyle={\small\ttfamily},
 keywordstyle=\bfseries,
}

\lstdefinestyle{customcpp}{
 aboveskip=0in,
 belowskip=0in,
 abovecaptionskip=0in,
 belowcaptionskip=0in,
 numbers=left,
 numberstyle=\tiny,
 %breaklines=true,
 captionpos=b,
 xleftmargin=\parindent,
 language=C++,
 %morekeywords={forall},
 showstringspaces=false,
 % dsm: semibold
 %basicstyle={\linespread{0.6}\fontseries{sb}\small\ttfamily},
 basicstyle={\fontseries{sb}\small\ttfamily},
 %basicstyle={\small\ttfamily},
 keywordstyle=\bfseries,
 commentstyle=\itshape\color{green!40!black},
}

% asf: stole this from HATS
\lstdefinestyle{custompython}{
 %aboveskip=0in,
 belowskip=0in,
 %abovecaptionskip=0in,
 belowcaptionskip=-10pt,
 %belowcaptionskip=1\baselineskip,
 breaklines=true,
 captionpos=b,
 language=Python,
 showstringspaces=false,
 numbers=left,
 stepnumber=1,
 % dsm: semibold
 basicstyle={\linespread{0.8}\fontseries{sb}\small\ttfamily},
 %basicstyle={\small\ttfamily},
 keywordstyle=\bfseries,
 %% columns=fullflexible,
 xleftmargin=2em,
 frame=single,
 framexleftmargin=2em,
 commentstyle=\itshape\color{green!40!black},
 morekeywords={to,yield},
}

%See: https://tex.stackexchange.com/questions/264361/skipping-line-numbers-in-lstlisting/
\let\origthelstnumber\thelstnumber
\makeatletter
\newcommand*\Suppressnumber{%
  \lst@AddToHook{OnNewLine}{%
    \let\thelstnumber\relax%
    \advance\c@lstnumber-\@ne\relax%
  }%
}
\newcommand*\Reactivatenumber[1]{%
  \setcounter{lstnumber}{\numexpr#1-1\relax}%
  \lst@AddToHook{OnNewLine}{%
    \let\thelstnumber\origthelstnumber%
    \refstepcounter{lstnumber}%
  }%
}

\hyphenation{timestamp time-stamp}
\hyphenation{Timestamp Time-stamp}
\hyphenation{timestamps time-stamps}
\hyphenation{Timestamps Time-stamps}

\newcommand{\sm}[1]{{\small #1}\xspace}
\newcommand{\app}[1]{{\texttt{#1}}\xspace}
\newcommand{\clopt}[1]{{\texttt{#1}}\xspace}
% victory: allow for quickly changing your mind on
%          whether you want parens after function names.
%\newcommand{\fnname}[1]{{\texttt{#1()}}\xspace}
\newcommand{\fnname}[1]{{\texttt{#1}}\xspace}
\newcommand{\varname}[1]{{\texttt{#1}}\xspace}
\newcommand{\instr}[1]{{\texttt{#1}}\xspace}

% Spacing
%\setlength{\textfloatsep}{8pt}
%\setlist[enumerate]{leftmargin=0.25in,itemsep=1pt,topsep=1pt}
%\setlength{\leftmargini}{0.125in}

% More compact paragraphs
\makeatletter
\renewcommand{\paragraph}[1]{\noindent {\bf #1}}
\makeatother

%\newcommand{\topbanner}{\textbf{DRAFT --- \input{auto_header.tex}}}
%\newcommand{\topbanner}{\textsc{Under Submission --- Please Do Not Distribute}}

% Headers -- comment these for submission
%\makeatletter
%\def\ps@plain{
%  \def\@oddhead{\hbox{}\normalsize\rightmark \hfil \topbanner \hfil}
%  \def\@evenhead{\hbox{}\normalsize\rightmark \hfil \topbanner \hfil}
%}
%\makeatother

% submission/web version
%\pagenumbering{arabic}

%\captionsetup[subfigure]{aboveskip=0pt,belowskip=-5pt}

% If you comment hyperref and then uncomment it, make clean first (or kill .aux files)
%\definecolor{lcolor}{RGB}{0, 56, 186} % {0, 35, 102}
%\hypersetup{bookmarks=true,breaklinks=true,letterpaper=true,colorlinks,linkcolor=black,citecolor=black,urlcolor=lcolor}
%\usepackage[pdfa]{hyperref}
\usepackage[pdfa,bookmarks=true,breaklinks=true,letterpaper=true,colorlinks,linkcolor=black,citecolor=black,urlcolor=black]{hyperref}


\renewcommand{\figureautorefname}{Figure\xspace}
% \newcommand{\subfigureautorefname}{Figure\xspace}
\renewcommand{\chapterautorefname}{Chapter\xspace}
\renewcommand{\sectionautorefname}{Section\xspace}
\renewcommand{\subsectionautorefname}{Section\xspace}
\newcommand{\algorithmautorefname}{Algorithm\xspace}
\renewcommand{\paragraphautorefname}{Section\xspace}
\renewcommand{\equationautorefname}{Equation\xspace}

% Temporary macros. Comment for submission!
\newcommand{\note}[1]{{\bf [~NOTE:~#1~]}}
\newcommand{\fixme}[1]{{\bf [~FIXME:~#1~]}}
\newcommand{\todo}[1]{{\bf [~TODO:~#1~]}}
%\newcommand{\tmp}[1]{{\textcolor{red}{#1}}}
\newcommand{\tmp}[1]{{#1}}
\newcommand{\OK}[1]{{#1}}

\renewcommand*{\ttdefault}{txtt}


%dsm: all these are retweaked
% Alter some LaTeX defaults for better treatment of figures:
% See p.105 of "TeX Unbound" for suggested values.
% See pp. 199-200 of Lamport's "LaTeX" book for details.
%   General parameters, for ALL pages:
\renewcommand{\topfraction}{0.9}        % max fraction of floats at top
\renewcommand{\bottomfraction}{0.8}     % max fraction of floats at bottom
%   Parameters for TEXT pages (not float pages):
\setcounter{topnumber}{5}
\setcounter{bottomnumber}{5}
\setcounter{totalnumber}{4}     % 2 may work better
\setcounter{dbltopnumber}{5}    % for 2-column pages
\renewcommand{\dbltopfraction}{0.9}     % fit big float above 2-col. text
\renewcommand{\textfraction}{0.07}      % allow minimal text w. figs
%   Parameters for FLOAT pages (not text pages):
\renewcommand{\floatpagefraction}{0.9}  % require fuller float pages
% N.B.: floatpagefraction MUST be less than topfraction !!
\renewcommand{\dblfloatpagefraction}{0.9}       % require fuller float pages

% Techniques
\newcommand{\name}{F1\xspace}
\newcommand{\noformatname}{SCC\xspace}

% Benchmarks. Make typos compile errors!
\newcommand{\bfs}{\app{bfs}}
\newcommand{\bfscage}{\bfs-\app{cage}}
\newcommand{\bfstric}{\bfs-\app{tric}}
\newcommand{\coloring}{\app{color}}
\newcommand{\mis}{\app{mis}}
\newcommand{\sssp}{\app{sssp}}
\newcommand{\kmeans}{\app{kmeans}}
\newcommand{\genome}{\app{genome}}
\newcommand{\des}{\app{des}}
\newcommand{\nocsim}{\app{nocsim}}
\newcommand{\silo}{\app{silo}}

\newcommand{\perlbench}{\app{perlbench}}
\newcommand{\bzip}{\app{bzip2}}
\newcommand{\gcc}{\app{gcc}}
\newcommand{\mcf}{\app{mcf}}
\newcommand{\milc}{\app{milc}}
\newcommand{\namd}{\app{namd}}
\newcommand{\gobmk}{\app{gobmk}}
\newcommand{\dealII}{\app{dealII}}
\newcommand{\soplex}{\app{soplex}}
\newcommand{\povray}{\app{povray}}
\newcommand{\hmmer}{\app{hmmer}}
\newcommand{\sjeng}{\app{sjeng}}
\newcommand{\libquantum}{\app{libquantum}}
\newcommand{\AVCref}{\app{h264ref}} % LaTeX commands cannot contain numbers :(
\newcommand{\lbm}{\app{lbm}}
\newcommand{\obmnetpp}{\app{omnetpp}}
\newcommand{\astar}{\app{astar}}
\newcommand{\sphinx}{\app{sphinx3}}
\newcommand{\xalancbmk}{\app{xalancbmk}}
\newcommand{\pricing}{\mcf}

\newcommand{\nnmcf}{\app{429.mcf}}
\newcommand{\nnmilc}{\app{433.milc}}
\newcommand{\nnhmmer}{\app{456.hmmer}}
\newcommand{\nnlibquantum}{\app{462.libquantum}}
\newcommand{\nnlbm}{\app{470.lbm}}
\newcommand{\nnastar}{\app{473.astar}}
\newcommand{\nnsphinx}{\app{482.sphinx3}}
\newcommand{\nnpricing}{\nnmcf}



\newcommand{\privalloc}{\app{privalloc}}

\usepackage{color}
\definecolor{tableaublue}{rgb}{0.44,0.62,0.81}
\definecolor{tableauorange}{rgb}{0.9,0.55,0.25} % original tableau value: {1.,0.62,0.29}
\definecolor{tableaugreen}{rgb}{0.4,0.75,0.36}
\definecolor{tableaured}{rgb}{0.93,0.4,0.36}
\definecolor{tableaupurple}{rgb}{0.68,0.55,0.79}
\newcommand{\green}[1]{\textcolor{tableaugreen}{\sf\bfseries #1}}
\newcommand{\orange}[1]{\textcolor{tableauorange}{\sf\bfseries #1}}
\newcommand{\blue}[1]{\textcolor{tableaublue}{\sf\bfseries #1}}
\newcommand{\red}[1]{\textcolor{tableaured}{\sf\bfseries #1}}
\definecolor{gray}{RGB}{128,128,128}
\newcommand{\gray}[1]{\textcolor{gray}{\sf\bfseries #1}}
%\newcommand{\purple}[1]{\textcolor{tableaupurple}{\sf\bfseries #1}}
% Taken from powerpoint, 40% lighter accent colors:
\definecolor{lightgreen}{RGB}{201,205,179}
\definecolor{lightblue}{RGB}{191,211,228}
\definecolor{lightorange}{RGB}{235,179,145}

\usepackage{pifont}
\usepackage{fourier-orns}
\definecolor{darkred}{rgb}{.65,0,0}
\definecolor{darkgreen}{rgb}{0,.5,0}
\definecolor{darkyellow}{rgb}{0.95,.6,0.1}
\newcommand{\cmark}{\normalsize \textcolor{darkgreen}{\ding{52}}}
\newcommand{\xmark}{\normalsize \textcolor{darkred}{\ding{56}}}
%\newcommand{\qmark}{\normalsize \textcolor{darkyellow}{\ding{115}}}%\ding{51}\hspace{-.0981in}\ding{55}}}
\newcommand{\qmark}{\normalsize \textcolor{darkyellow}{\decofourleft}}%\ding{51}\hspace{-.0981in}\ding{55}}}
\newcommand{\badyes}{\normalsize \textcolor{darkred}{$\bullet$}}
\newcommand{\goodno}{}
\newcommand*{\thead}[1]{%
\multicolumn{1}{c}{\begin{tabular}{@{}c@{}}#1\end{tabular}}}
\newcommand*{\theadbf}[1]{%
\multicolumn{1}{c}{\bfseries\begin{tabular}{@{}c@{}}#1\end{tabular}}}
